\documentclass{article}


\begin{document}

\section{Glossary}
\begin{itemize}
    \item Explanatory variable (input variable, predictor, covariate): Often written as $x$. Generally written as $x$ if the data has been observed, $\tilde{x}$ if it hasn't.
    \item Outcome variable (output variable, target, response): Often written as $y$. Generally written as $y$ if the data has been observed, $\tilde{y}$ if it hasn't.
    \item Data: The combination of input and output variables that have been observed. $D = {(x_i, y_i)}$.
    \item Model: Some way (parametric or non) of relating the input and output variables. $y = M(x, \theta)$.
    \item Likelihood: The liklihood of the data given the model (M), its parameters ($\theta$); $\mathcal{L}(D | M, \theta)$
    \item Prior probability distribution: The probability of the parameters ($\theta$) for the model, before conditioning on the data. This maybe be determined by previous work, or natural limit (some properties, such as variance, must be $> 0$).
    \item Posterior probability distribution: The probability of the parameters ($\theta$) for the model, given the data. $p(\theta | D, M)$
    \item Prior predictive distribution: The prediction if $y$ for $x$, given a model and a prior distribution over the parameters for that model. $p(y | x) = \int p(y | x, \theta) p(\theta) d\theta$.
    \item Posterior predictive distribution: The prediction of $\tilde{y}$ for $\tilde{x}$, given a model conditioned on some data; $p(\tilde{y} | \tilde{x}, D, M)$. In the process of conditioning on the data, we obtain a posterior on the model parameters. Thus, $p(\tilde{y} | \tilde{x}, D, M) = \int p(\tilde{y} | \tilde{x}, \theta, M) p(\theta | D, M) d\theta$, or the probaility of $\tilde{y}$ at a given $\theta$, weighted by the posterior probability of that $\theta$.
    \item Latent variable
\end{itemize}

Let's give a full example. Consider subhalo abundance matching, a method of populating dark matter halos with stellar mass. Our target, $y$, is the stellar mass. We choose an explanatory variable (such as halo mass) that we thing correlates well (covaries) with this - this is $x$.
We have some model $M$ to compute $y | x$. This model parameterizes the SMHM relation and its scatter $y = M(x, \theta)$.
To find those other parameters, we find the liklihood of our data given the model and $\theta$. To do this, we also assume a cosmology and that an N-body simulation is a good realization of the universe. The Likelihood compares the summary statistic of the N-body + model and observed data.

\end{document}
