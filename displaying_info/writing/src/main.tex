\documentclass{article}

\usepackage{xcolor}
\usepackage{qtree}

% \newcommand{\hl}[1]{\colorbox{green}{#1}}
\newcommand{\hl}[1]{\underline{#1}}
\newcommand{\hlq}[1]{\colorbox{red}{#1}}
\newcommand{\example}[1]{\textit{#1}}


\begin{document}

\section{Grammar}


\subsection{Some Definitions}

Predicate: The verb like part of a clause.
\example{I \hl{listened to him talk}}

\subsection{Clauses}

A group of words with both subject and predicate. Divided into a few categories:

Independent Clause: A group of words that could stand on its own as a sentence. It is a complete thought.
\example{\hl{I am learning to write.}}


Dependent Clause: A group of words that is not a complete thought. If this comes before the independent clause it is usually separated with a comma.
\example{\hl{Because I write poorly at the moment}, I am writing this document.}

Relative Clause: A clause usually introduced by the relative pronouns who, whom, which or whose, by the subordinator that and by the relative adverbs where, why and when


Restrictive (or defining, integrated relative) clause: A clause that functions as an adjective to identify the word it modifies. Note that this doesn't just provide extra information. It is essential to understanging. It is not offset by commas.
\example{I wrote this paper with the professor \hl{who is my advisor}}
\example{I fed the dog \hl{that I found outside}}

Nonrestrictive (or non-defining, non-integrated relative) clause: This is also an adjectival clause, though it contains non-essential information. It is offset by commas.
\example{Bob, \hl{who I met last week}, is a nice guy}


\subsection{Phrases}

A group of words that doesn't have both subject and object. Often in one of these forms:

Prepositional Phrase:
\example{After meeting with him}


\subsection{Syntax}

Syntax is the set of rules that govern the structure of sentences. Or, knowledge of syntax is what allows us to translate our amorphous ideas into a tree of phrases and then into a string of words. Consider.

\example{In Sophocles' play, Oedipus marris his mother}

This breaks down into,

\Tree[.Clause
    [.Prepositional\ Phrase
        [.Preposition \textit{In} ]
        [.Noun\ Phrase \textit{Sophocles' play} ]
    ]
    [.Clause
        [.Noun\ Phrase \textit{Oedipus} ]
        [.Verb\ Phrase \textit{married his mother} ]
    ]
]

We could have broken this down into more detail (\textit{married his mother} is very followed by a noun phrase), but I think this gives the main idea.


\section{Ordering}

There are many ways to say the same thing.
\example{Bob flew the helicopter over the river},
\example{The helicopter was flown by Bob over the river}
These are two common ``voices'', active and passive. There are more.


\subsection{Active vs Passive}

In the active voice, sentences are ordered subject-verb-object.
\example{I ate the pizza.}
In the passive voice, the order becomes object-<to be>-verb-subject.
\example{The pizza was eaten by me.}
The subject no longer acts.

In what cases might we want to use the passive?
\begin{itemize}
    \item We don't know (or care about) the subject. In the passive voice, the ``by-phrase'' can be omitted and so we can write \example{The man was pulled from the burning house} rather than \example{The fireman pulled the man\ldots}.
    \item The object connects more readily with the previous sentence. Compare \example{I brought my lunch to work. It was made last night by my favourite indian restaurant.} with \example{I brought my lunch to work. My favourite indian restuarant made it last night}. After the first sentence, ``my lunch'' is on our minds. If there is anything new to be said about the food, we should begin by mentioning it. In the second example, the second sentence begins with the digression ``My favourite indian restaurant'', and only later connects it back the sentence before. This adds cognitive load.
    % \item Relative clauses should generally be passive. \example{Among those at the meeting was Bob, the former helicopter pilot chosen by his boss Brian to represent him, and Tom}. \example{Among those at the meeting was Bob, the former helicopter pilot who was chosen by his book Brian to represent him, and Tom}
\end{itemize}

\subsection{Topic, then comment. Given, then new}

People learn by integrating new information into their existing web of knowledge. If a fast is presented without any background or link to what came before, the reader will have to hold this fact in memory while waiting for context. Don't force them to do this! First present the context. Often the only way to do this is to use the passive voice (as English requires subject - verb ordering).



\subsection{Light before Heavy}

The US constitution says ``among these are life, liberty and the pursuit of happiness''. Why does this sound so much better than ``among these are the pursuit of happiness, liberty and life''?







\section{Punctuation}

\subsection{Comma}

Two main purposes:
First, to separate parenthetical comments about a thing (action, event, state), from the words necessary to pin down the thing itself. Second, to signal a break in pronounciation (sometimes called a prosodic break).


To illustrate the first, compare:
\example{Sticklers who don't understand the conventions of punctuation shouldn't criticize errors by others.}
\example{Sticklers, who don't understand the conventions of punctuation, shouldn't criticize errors by others.}

In the first of these we single out a subset of sticklers with a restrictive clauses. The second makes a parenthetical jibe at sticklers and so this is separated.

Consider also,
\example{I visited my friend Bob}
\example{I visited my friend, Bob}

In the first, it is important that I visited Bob. In the second, it is only significant that I visited a friend (and by the way, the friend's name is Bob).

After an introductory phrase.
\example{After meeting with him, I went to lunch}



Then why are commas so complicated? They serve two purposes: they both signal a syntactic break (separating a phrase or clause) and a prosodic break, a slight pause in pronunciation.
Often these line up, but sometimes they don't and that causes problems. Consider,
\example{When the supplementary phrase is short\hlq{,} a speaker naturally skates right over it to the next phrase in the sentence, and it is then unclear whether the punctuation should reflect the syntax or the sound.}
While a comma that demarcates a syntactic break can be ommited when the pronunciation flows through it, the converse is not allowed: a comma may not separate the elements of an integrated phrase.

\subsection{Semicolon}

\subsection{Colon}


\end{document}


https://www.grammarly.com/blog/comma/
https://www.grammar-monster.com/glossary/restrictive_clause.htm
https://www.k12reader.com/term/nonrestrictive-clause/
https://caxton1485.wordpress.com/2014/04/11/grammar-basics-relative-clauses/
