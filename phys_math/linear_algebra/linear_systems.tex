\documentclass{article}
\usepackage{amsmath}
\usepackage{hyperref}


\begin{document}
\noindent\makebox[\textwidth][c]{\Large\bfseries Linear Systems}

\section{Basics}

Consider the system of,

\begin{align}
    y_1 &= 2x_1 + x_2 \\
    y_2 &= 2x_2
\end{align}

\noindent This can be written as,

\begin{align}
    y &= Ax \\
    A &=
    \begin{bmatrix}
        2 & 1 \\
        0 & 2
    \end{bmatrix}
\end{align}

\section{Differential Equations}

Consider the system of,

\begin{align}
    \dot{x} &= y \\
    \dot{y} &= -x
\end{align}

\noindent which as above is (writing $x$ as a vector now, where $x_1 = x$ and $x_2 = y$ above),

\begin{align}
    \dot{x} &= Ax \\
    A &=
    \begin{bmatrix}
        0 & 1 \\
        -1 & 0
    \end{bmatrix}
\end{align}

\noindent The solution to differential equations of the form $\dot{x} = Ax$ is $x = Ce^{At}$ where $C$ is some constant.
This is true even when A is a matrix, though in this case $C$ becomes a vector.
The value of this exponent is found from the sum of an \href{https://en.wikipedia.org/wiki/Matrix_exponential}{infinite power series}.

\begin{equation}
    e^{At} = \sum_{k=0}^{\infty} \frac{(At)^k}{k!} = I + At + \frac{(At)^2}{2} + \ldots{}
\end{equation}

\noindent This is not a special case for the exponentiation of a matrix, but just an extension of the definition of the \href{https://en.wikipedia.org/wiki/Exponential_function#Formal_definition}{exponential function}.

\begin{equation}
    e^a = \sum_{k=0}^{\infty} \frac{a^k}{k!} = 1 + a + \frac{a^2}{2} + \frac{a^3}{6} + \ldots{}
\end{equation}

\noindent for the case of $A$ above, $e^{At}$ evaluates to,

\begin{equation}
    \begin{bmatrix}
        \cos(t) & \sin(t) \\
        -\sin(t) & \cos(t)
    \end{bmatrix}
\end{equation}

\noindent and so we have

\begin{equation}
    x = \begin{bmatrix}
        \cos(t) & \sin(t) \\
        -\sin(t) & \cos(t)
    \end{bmatrix}
    \begin{bmatrix}
        C_1 \\
        C_2
    \end{bmatrix}
\end{equation}

\noindent How do we choose $C$? For that we need the initial conditions. Let's say that,

\begin{equation}
x(0) \equiv x_0 =
    \begin{bmatrix}
        1 \\
        2
    \end{bmatrix}
\end{equation}

\noindent This gives us the set of equations,

\begin{align}
    x_1 &= C_1 \cos(t) + C_2 \sin(t) = 1 \\
    x_2 &= C_2 \cos(t) - C_1 \sin(t) = 2
\end{align}

\noindent which at $t = 0$ gives us, $C_1 = 1$, $C_2 = 2$. So,

\begin{equation}
    x = \begin{bmatrix}
        \cos(t) & \sin(t) \\
        -\sin(t) & \cos(t)
    \end{bmatrix}
    \begin{bmatrix}
        1 \\
        2
    \end{bmatrix}
\end{equation}

\end{document}
