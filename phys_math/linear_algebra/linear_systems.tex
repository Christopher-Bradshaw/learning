\documentclass{article}
\usepackage{amsmath}
\usepackage{hyperref}


\begin{document}
\noindent\makebox[\textwidth][c]{\Large\bfseries Linear Systems}

\section{Basics}

Consider the system of,

\begin{align}
    y_1 &= 2x_1 + x_2 \\
    y_2 &= 2x_2
\end{align}

\noindent This can be written as,

\begin{align}
    y &= Ax \\
    A &=
    \begin{bmatrix}
        2 & 1 \\
        0 & 2
    \end{bmatrix}
\end{align}

\section{Differential Equations}

Consider the system of,

\begin{align}
    \dot{x} &= y \\
    \dot{y} &= -x
\end{align}

\noindent which as above is (writing $x$ as a vector now, where $x_1 = x$ and $x_2 = y$ above),

\begin{align}
    \dot{x} &= Ax \\
    A &=
    \begin{bmatrix}
        0 & 1 \\
        -1 & 0
    \end{bmatrix}
\end{align}

\noindent The solution to differential equations of the form $\dot{x} = Ax$ is $x = e^{At}$.
This is true even when A is a matrix. However, the value of this exponent is found from the sum of an \href{https://en.wikipedia.org/wiki/Matrix_exponential}{infinite power series}.

\begin{equation}
    e^A = \sum_{k=0}^{\infty} \frac{1}{k!} A^k
\end{equation}

\noindent This reduces to the expected $e^a$ when $A$ is a $1 \times 1$ matrix with value $a$. See the \href{https://en.wikipedia.org/wiki/Exponential_function#Formal_definition}{exponential function}.

\begin{equation}
    e^a = \sum_{k=0}^{\infty} \frac{a^k}{k!} = 1 + a + \frac{a^2}{2} + \frac{a^3}{6} + \ldots{}
\end{equation}

\end{document}
