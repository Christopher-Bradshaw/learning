\documentclass{article}
\usepackage{amsmath}
\usepackage{hyperref}

\newcommand{\norm}[1]{\ensuremath{\left\vert #1 \right \vert}}

\begin{document}
\noindent\makebox[\textwidth][c]{\Large\bfseries Three Body Problem}

\section{Newtonian Formulation}

We can use Newton's second law to set up a system of differential equations,

\begin{align}
    F_{\text grav} &= \frac{GMm}{r^2} \hat{r} \\
    m\ddot{r} &= \frac{GMm}{r^2} \hat{r} \\
\end{align}

\noindent Which for each body results in something like this,

\begin{equation}
    \ddot{r}_1 = -Gm_2\frac{r_1 - r_2}{\norm{r_1 - r_2}^3} - Gm_3\frac{r_1 - r_3}{\norm{r_1 - r_3}^3}
\end{equation}

\noindent Giving us 9 second order differential equations; 1 for each component ($x, y, z$) of each of the three bodies.


We can instead choose to work with 18 first order differential equations by adding a velocity like variable, $v = \dot{r}$.

\begin{align}
    \dot{r}_1 &= v_1 \\
    \dot{v}_1 &= -Gm_2\frac{r_1 - r_2}{\norm{r_1 - r_2}^3} - Gm_3\frac{r_1 - r_3}{\norm{r_1 - r_3}^3}
\end{align}

\section{Hamiltonian Formulation}

In practice, the 18 differential eqautions we use are from the Hamiltonian, $\mathcal{H}$.

\end{document}
