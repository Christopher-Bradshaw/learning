\documentclass{article}
\usepackage{amsmath}
\usepackage{hyperref}

\newcommand{\norm}[1]{\ensuremath{\left\vert #1 \right \vert}}
\newcommand{\vb}[1]{\ensuremath{\mathbf{#1}}}

\begin{document}
\noindent\makebox[\textwidth][c]{\Large\bfseries Three Body Problem}

\section{Newtonian Formulation}

We can use Newton's second law to set up a system of differential equations,

\begin{align}
    F_{\text grav} &= \frac{GMm}{r^2} \hat{r} \\
    m\ddot{r} &= \frac{GMm}{r^2} \hat{r}
\end{align}

\noindent Which for each body results in something like this,

\begin{equation}
    \vb{\ddot{r}_1} = -Gm_2\frac{\vb{r_1} - \vb{r_2}}{\norm{\vb{r_1} - \vb{r_2}}^3} - Gm_3\frac{\vb{r_1} - \vb{r_3}}{\norm{\vb{r_1} - \vb{r_3}}^3}
\end{equation}

\noindent Giving us 9 second order differential equations; 1 for each component ($x, y, z$) of each of the three bodies.

\begin{equation}
    \ddot{x}_1 = -Gm_2\frac{x_1 - x_2}{\norm{\vb{r_1} - \vb{r_2}}^3} - Gm_3\frac{x_1 - x_3}{\norm{\vb{r_1} - \vb{r_3}}^3}
\end{equation}

We can instead choose to work with 18 first order differential equations by adding a velocity like variable, $v = \dot{r}$.

\begin{align}
    \dot{r}_{x, 1} &= v_{x, 1} \\
    \dot{v}_{x, 1} &= -Gm_2\frac{x_1 - x_2}{\norm{\vb{r_1} - \vb{r_2}}^3} - Gm_3\frac{x_1 - x_3}{\norm{\vb{r_1} - \vb{r_3}}^3}
\end{align}

\section{Hamiltonian Formulation}

In practice, the 18 differential eqautions we use are from the Hamiltonian, $\mathcal{H}$.

\begin{equation}
\mathcal{H} = - \frac{G m_1 m_2}{\norm{\vb{r_1} - \vb{r_2}}}
                - \frac{G m_1 m_3}{\norm{\vb{r_1} - \vb{r_3}}}
                - \frac{G m_2 m_3}{\norm{\vb{r_2} - \vb{r_3}}}
                + \frac{\vb{p_1^2}}{2 m_1}
                + \frac{\vb{p_2^2}}{2 m_2}
                + \frac{\vb{p_3^2}}{2 m_3}
\end{equation}

\noindent which is the total mechanical energy in the system (potential + kinetic). The relevant equations in this formulation are,

\begin{align}
    \vb{\dot{r}_1} &= \frac{d \mathcal{H}}{d \vb{p_1}} = \vb{v_1} \\
    \vb{\dot{p}_1} &= - \frac{d \mathcal{H}}{d \vb{r}_1} = \frac{G m_1 m_2}{ \norm{\vb{r_1} - \vb{r_2}}^2 }\\
\end{align}

\end{document}
